\documentclass[]{interact}

\usepackage{epstopdf}
\usepackage{subfigure}

%\usepackage[nolists,tablesfirst]{endfloat}% To `separate' figures and tables from text if required

%\usepackage[longnamesfirst,sort]{natbib}
%\bibpunct[, ]{(}{)}{;}{a}{,}{,}
%\renewcommand\bibfont{\fontsize{10}{12}\selectfont}

\usepackage[natbibapa]{apacite}
%\usepackage{apacite}
\setlength\bibhang{12pt}
\renewcommand\bibliographytypesize{\fontsize{10}{12}\selectfont}

\setlength\parindent{0pt}
\setlength\parskip{8pt}

\theoremstyle{plain}
\newtheorem{theorem}{Theorem}[section]
\newtheorem{lemma}[theorem]{Lemma}
\newtheorem{corollary}[theorem]{Corollary}
\newtheorem{proposition}[theorem]{Proposition}

\theoremstyle{definition}
\newtheorem{definition}[theorem]{Definition}
\newtheorem{example}[theorem]{Example}

\theoremstyle{remark}
\newtheorem{remark}{Remark}
\newtheorem{notation}{Notation}

\begin{document}

%\articletype{ARTICLE TEMPLATE}

\title{
 The Relationship between Bloom's Cognitive Levels and Question Difficulty:
   An Investigation of Question Dependencies
}

\author{
\name{D. G. Castleberry and S. R. Brandt}
\affil{Center for Computation and Technology, Louisiana State University, Baton Rouge}
}

\maketitle

\begin{abstract}
  This paper explores the relationship between Bloom's cognitive levels and
  difficulty in an intelligent tutoring system.  First, it addresses the
  question of whether or not a distinction can be made between Bloom taxonomic
  levels and question difficulty.  Second, given that a distinction can be
  made, it explores the possiblity of modelling the probability of student
  success on a given question based on its parameters and those past answered
  questions.  Third, it investigates the potential to increase that probability
  by asking ``intervention questions''. 
\end{abstract}

\begin{keywords}
Bloom's Taxonomy, Item Response Theory, difficulty
\end{keywords}

\section{Introduction}

\subsection{Bloom's Taxonomy}

Bloom's cognitive taxonomy organizes questions into levels depending on the
cognitive functions required of the answerer.  The levels are: knowledge,
comprehension, application, analysis, evaluation, and synthesis.  A brief
overview is given below, with definitions and examples of questions covering
the concept of for-loops:

\begin{itemize}

\item \textbf{Knowledge}. Recalling factual information.  \emph{What is a
for-loop?}

\item \textbf{Comprehension}. Assigning meaning to information.  \emph{What
does the example for-loop output? (Give example.)}

\item \textbf{Application}. Applying a rule to a specific instance.  \emph{How
can the update statement of the loop be changed to print only even numbers?}

\item \textbf{Analysis}. Breaking information into parts and exploring
relationships.  \emph{What would happen if the update statement decremented
instead of incremented the counter?}

\item \textbf{Evaluation}. Judging the use of knowledge or the validity of an
argument.  \emph{Which is better for reading user input: a for-loop or a
while-loop? Why?}

\item \textbf{Synthesis}. Utilizing knowledge to create a new solution to
satisfy a goal.  \emph{Write a for-loop to print only even numbers up to ten.}

\end{itemize}

Each category depends on the cognitive functions used in the previous category.
That is, comprehension requires knowledge, application requires comprehension,
and so forth.  Furthermore the mastery of one of the levels is with respect to
a given concept.  A student may be able to synthesize solutions to problems
dealing with expressions, but may not possess knowledge of equations, and thus
could not solve problems involving equations.

The utility of Bloom's taxonomy lies in its ability to pinpoint the underlying
cause of the student's problem-solving impasses \citep{shuhidan2011}.  Suppose a
test of mastery of loops is given with the comprehension question ``What does
such-and-such loop output?'' is given, and the student reaches an impasse.  If
the question ``What are the three expressions of the loop and what do they
do?'' is asked and the student does not know, the impasse can be attributed to
a lack of knowledge about loops.  If the student does know about the loop
expressions but still cannot answer, one might instead attribute it to a
comprehension difficulty \emph{as such}; which might be remedied by giving some
examples to build intuition, then continuing to test at the comprehension
level.

Educators may have an intuitive notion of how to do this, but Bloom's taxonomy
gives the ability to examine the impact of questions scientifically.  By
identifying the tested concept and the Bloom level of exercises, one can then
form hypotheses about student responses to questions.  


\subsection{Item Response Theory}

We now introduce a mature assessment theory known as Item Response Theory, an
alternative to Classical Test Theory (CCT).  Whereas Classical Test Theory
assigns a student a grade based on the student's position in a distribution of
composite test scores, Item Response Theory accounts for item difficulty,
item discrimination, the probability of guessing the question correctly.

One of the main appeals of IRT is its incorporation of item difficulty.
As we will see later, this is pertinent to our interpretation of Bloom
levels.  

According to Item Response Theory (IRT), the probability $p_i$ that a student
answers correctly the $i^{th}$ question on a test, is given by:

\[
  p_i(\theta) = \gamma + \frac{1-\gamma_i}{1+e^{\alpha_i(\theta-\beta_i)}}
\]

where:

\begin{itemize} 

 \item $\alpha$ is the item discrimination, or how well the item can
 distinguish students of varying trait ability;

 \item $\beta$ is the question difficulty, an initial estimate of which can be
 obtained from the proportion of students with average trait ability who pass
 the question;

 \item $\gamma$ is the probability of guessing the answer correctly,
 which for $n$-choice questions is $1/n$;

 \item and $\theta$ is the \emph{trait ability} of the student, or the
 student's particular ability to answer that question correctly;

\end{itemize} 

Trait ability in IRT can be obtained using the Newton-Raphson method, which
finds a maximum-likelihood estimate of $\theta$ \citep{baker2004}.  The utility
of IRT is that it requires fewer questions to gauge trait ability due to its
account of other factors ($\alpha, \beta, \gamma$). It is thus reputed to be a
more mature means of assessment than CTT.

Until now, the fusion of IRT and Bloom's taxonomy has only existed in the
literature as a possibility \citep{sitthisak}.  This work seeks to reconcile
the two by offering a compatible interpretation of Bloom's taxonomy.

\subsection{Previous Work}

Bloom's taxonomy has been proven to be useful at the undergraduate level, and
particularly in the field of software engineering \citep{britto2015,
mahmood2014}.  It has seen success in program comprehension
\citep{buckley2003}, where the asking of comprehension questions fosters code
reading \citep{losada2008}. In addition, it has been useful for pinpointing the
difficulties of novice programmers in a guided learning approach
\citep{shuhidan2011}.  It been used to identify a marked preference for
higher-level problems for those able to solve them \citep{bruyn2011}
\citep{goel2004}.  At least two experiments have shown the effect of item
ordering on performance \citep{newman1988effect,castleberry2016effect}, and the
taxonomy has even been applied to create ratings of courses based on the
average Bloom levels of tasks and questions in the course
\citep{oliver2004course}.

In spite of all this, there is an ongoing debate regarding the applicability of
Bloom's taxonomy to computer science \citep{johnson2006bloom,
fuller2007developing, thompson2008bloom}.  The crux of this debate centers
around the interpretation of Bloom levels: not only how questions map to Bloom
levels, but also regarding the progression of Bloom levels over the span of a
course or curriculum.  A tacit assumption in much of the research is that Bloom
levels equate to difficulty levels.  The current research seeks to open a
dialogue about this interpretation in particular.

We have designed an intelligent tutoring system to explore the effect of
question difficulty as it pertains to Bloom's taxonomy.  The system is based on
previous research on executable paper systems (for generating publication
manuscripts) which use R as an underlying framework \citep{castleberry2011}.
Our intelligent tutoring system is available for browsing
on-line\footnote{https://steam.cct.lsu.edu/}. It currently supports short
answer and multiple choice formats.

\section{Hypotheses}

\subsection{Preliminary Discussion}

A popular interpretation of Bloom's cognitive levels is that they are
difficulty levels, and that these difficulty levels are fixed
\citep{newman1988effect,oliver2004course,lord2007moving,
johnson2006bloom,fuller2007developing}.  Knowledge is easy; synthesis is hard.
We will call this the Bloom-equals-difficulty hypothesis.

A creative interpretation of this hypothesis is that per-student difficulty can
be explained in terms of the demotion of Bloom levels.  The synthesis \emph{and
therefore difficult} questions, once the solution is obtained, are reduced to
knowledge \emph{and therefore easy} questions.  Certainly, this is one possible
scenario.  However, it does not explain a student's ability to answer
altogether distinct synthesis problems; this is to suggest students engage the
cognitive functions of synthesis, presumably as an effect of learning the use
of those functions.  To maintain that Bloom equals difficulty would either
require sustaining the view that students are able to pass the questions
because of a demotion of the cognitive functions required (which poses a rather
cynical view of cognition and of education); or else some notion of trait
ability--in particular one which offsets difficulty, allowing an increase in
the probability of passing the question.

The latter option takes the shape of a primeval form of IRT, mapping Bloom
levels to $\beta$.  This is highly convenient for the hypothesis that Bloom
equals difficulty, as it becomes agreeable to evaluation with IRT.  Nontheless,
there are still several issues with the hypothesis:

% Ideal mapping


Consider first that exposure to knowledge and knowledge questions causes an
increase in $\theta$, provided the student acquires the knowledge.  According
to Bloom's theory, this would increase the probability of answering
comprehension questions correctly.  This agrees with the theory.  However,
according to IRT, it would also increase the probability of answering
application questions correctly even if no comprehension-level questions are
asked.  In fact, it is theoretically possible to design tests with sufficient
numbers of knowledge questions to place $\theta-2$ much greater than zero--that
is to say, asking a sufficient number of knowledge questions should give the
reasonable assurance that application-level ability is high enough not to
bother testing it.

As many educators will no doubt agree, this is not the case. Even if a student
scores exceptionally well on knowledge questions, it gives no such assurance
that the student will score similarly on application-level questions.  Even if
knowledge and application scores are found to be correlated, there may be an
additional underlying factor contributing to both scores (such as
intelligence).  The application level is \emph{qualitatively} different; hence
it needs to be tested in spite of the value of $\theta$.

Second, if the hypotheses is strictly true, then there can exist no
counterexample to the claim.  That is to say that there exists no problem on a
lower Bloom level which is more difficult than a problem on a higher Bloom
level.  To take an example, it would be mistaken to call an application problem
easier than any comprehension problem. 

Counterexmaples with intuitive appeal can be constructed.  Take for-loops for
example.  To ask a student to print out the result of a simple loop which
prints numbers 1 to 10 should present an easy enough task.  This is an
application of the many rules of for-loops, but which does not require creative
synthesis or critical thinking on the part of the student (hence it is an
application problem). 

Consider then asking the student to impose a flowchart over the same code,
including the initialization, condition and update statements and the body of
the loop, and to indicate the start and stop of the loop.  Suppose the student
has seen flowcharts over similar loops before.  This is a test of
comprehension--in particular, comprehension of the internal workings (the
control flow) of the loop.    

The application problem above requires only an intuitive comprehension of the
loop: ``\emph{It prints the sequence from 1 up to 10 in steps of 1}'', the
student may realize.  It demands only shallow comprehension and the rule to be
applied is simple.  In the comprehension problem, on the other hand, a higher
degree of precision in comprehension of the loop is demanded to solve the
problem.  The comprehension problem also depends on mastery of the knowledge,
comprehension, and application of flowchart symbols.  Regardless, explaining
the control flow of the loop in full detail is a more difficult undertaking
than simply printing its output. 

Likewise, it is possible to ask the student to synthesize a loop printing the
first ten powers of two, then ask a comparatively difficult analysis question
in the form of an obfuscated loop code.  Impasses in this situation may be
attributed to a lack of knowledge or comprehension of the constructs used in
the presented code.  In the synthesis problem, the student has the advantage
of using known syntax. 

Third, by Bloom's own admission, it is possible to skip levels for certain
concepts \citep{bloom1956}.  In this case, the ordinality of Bloom levels
breaks down.  A plausible example is given above.  It is possible to teach how
to obtain the numeric sequences printed by for-loops before covering in full
detail the control flow of the loop.  In that particular case, application of
the rule does not depend on comprehension of the code construct.

Alternative interpretations of Bloom levels distinguish between facility or
difficulty and complexity of a problem \citep{hill1981testing,
thompson2008bloom}.  In such interpreations, the cognitive complexity has an
orthogonal relationship to the item difficulty.  The problem with such
interpretations, however, is that they do not readily explain the moderate
correlation that does exist between mean performance and Bloom level
\citep{hill1981testing}.

With this in mind, we would like to introduce an alternative to the hypothesis
that Bloom levels equate to difficulty.  We posit that Bloom levels indicate
\emph{dependency relationships}.  As the theory goes, knowledge is a
\emph{requirement} for comprehension; but this is not to say that the
acquisition of the knowledge is somehow easier.  Our vision of dependency
relationships places the items in a directed graph, such that an edge pointing
out from a problem A to B represents a dependency which A has on B.

To evaluate our hypothesis, we introduce the notion of an intelligent tutoring
system.  An intelligent tutoring system is a computer-based system which
interacts with a student with the end-goal of teaching material, but adapts its
interaction in some way to the student's responses, much as a human tutor
would.  With such a system, we can automate the scheduling of questions,
collect responses, and analyze them. For the purpose of evaluating our
hypothesis, we can associate difficulty and Bloom levels with questions, then
collect responses to test the hypothesis statistically.

\subsection{Research Questions}

Here we clarify our specific research questions:

\textbf{Question 1}. Is there a difference between Bloom level and difficulty?
(That is, can an appreciable amount of variance in student responses be
attributed to Bloom level?)

\textbf{Question 2}. Given that there exists a difference between Bloom level
and difficulty, is it possible to modify IRT to more accurately predict
$p(x_i)$ given Bloom level information?  That is, is it possible to reconcile
Bloom's taxonomy and IRT using our newfound interpretation of Bloom levels?

\textbf{Question 3}. Given a predictive model and parameters for a question
$x_i$, is it possible to manipulate a target probability $p(x_i)$ by asking
related questions?  In particular, can we use our model to ease a student's
transition to higher Bloom levels?

\section{Methodology and Analysis}

Our intelligent tutoring system has a bank of questions which are tagged by
Bloom level, concept, and question difficulty.  We can use our system to easily
create groups of questions (which we call assessments) and issue them either on
paper or electronically.  Our intelligent tutoring system supports short-answer
and multiple-choice formats, and can provide immediate feedback to the student
on the correctness of their response.

\subsection{Experiment 1}

\textbf{Experiment 1 Design}. In the first experiment, a selection of 16
multiple-choice questions on topics in operating systems concepts were asked to
a sample group of students (N=54) in an undergraduate-level operating systems
class.  Each question had 5 choices ($\gamma$=.2). These questions were tagged
with Bloom levels, and interrater reliability was calculated between two
independent raters with experience in curricular and course design.  Questions
were tagged with hypothesized difficulty levels, which were targeted to be
inversely proportional to the Bloom levels.  Hence the question author intended
for the knowledge questions to be difficult relative to analysis questions,
while still maintaining the appropriate Bloom category.


\textbf{Analysis of Experiment 1 Data}.  Interrater reliability on Bloom levels
was calculated to be 93\%.  Means and standard deviations for each Bloom level
are reported below.

\begin{center}
\begin{tabular}{|l|l|l|l|l|}
                                                                 \hline
      &   Knowledge & Comprehension & Appplication & Analysis \\ \hline
 $M$  &  .82 & .67 & .53 & .42  \\ \hline
$SD$  &  .12 & .11 & .09 & .09  \\ \hline
\end{tabular}
\end{center}

In light of this data, a modifiction of IRT was sought in order to account for
the dependency relationships that existed among the questions.  Intuitively,
this modification of $p_i$ should possess the properties that: (a) if a
question $x_j$ is unrelated, $p_i$ should be unaffected; and (b) if $x_j$ is
fully related, $p_i$ should be equal to the response value for $x_j$. 

The degree of relationship may be expressed with Pearson's $r$, where
$r_{ij}=0$ indicates no relationship between $x_i$ and $x_j$ and $r=1$
indicates a full positive correlation.  An equally viable metric is the
coefficient of determination $r^2$, which indicates the proportion of variance
in the depender which is attributable to the dependee.  Intuitively, it is
this proportion of the probability which should be associated with the
correctness of $j$.  The remainder may be left to item response theory.
In the event of a negative correlation, the response should be opposite.


\begin{align*}
  p_i(\theta) &=& (1-r^2) & \Big(\gamma_i + \frac{1-\gamma_i}{1+e^{\alpha_i(\theta-\beta_i)}}\Big) \\
              &+& (r^2)   & \Big(\mathbf{sgn}(r)x_{sj}+\frac{1-\mathbf{sgn}(r)}{2}\Big)
\end{align*}

The above formula satisfies our needs.  For $r=1$, $p_i$ defaults to the
correctness of the dependee's answer (0 or 1, depending on whether or not the
dependee was answered correctly);  and for $r=-1$, $p_i$ assumes the flip of
the dependee's answer.  For $r=0$, $p_i$ assumes the form of original IRT.  

\subsection{Experiment 2}

\textbf{Experiment 2 Design}.  To test this modified theory, another experiment
was conducted.  The purpose of this experiment was to explore the relationship
among the probability predicted by unmodified IRT, the probability predicted by
modified IRT, and the correctness of the actual response.  It was hypothesized
that modified IRT should produce a more accurate output than unmodified due to
its account of dependency relationships.

As in the first experiment, a selection of 16 multiple-choice questions on
topics in operating systems concepts were asked to the same sample group of
students (N=54) in the same manner ($\gamma$=.2).  The questions covered 4
concepts, and each concept had 4 questions, each of different Bloom levels
(Knowledge, Comprehension, Application, Analysis), where each question of a
Bloom level higher than Knowledge was dependent on a question of an
immediately-lower Bloom level (e.g. the Application question was dependent on
the Comprehension question of the same concept).


\textbf{Analysis of Experiment 2 Data}. \ldots Interrater reliability on Bloom
levels was once again calculated to be 93\%. 

\begin{center}
\begin{tabular}{|l|l|l|l|l|}
                                                                 \hline
      &   Knowledge & Comprehension & Appplication & Analysis \\ \hline
 $M$  &  .78 & .62 & .48 & .41  \\ \hline
$SD$  &  .19 & .12 & .12 & .08  \\ \hline
\end{tabular}
\end{center}

To test the hypothesis that modified IRT leads to an increase in accuracy,
a t-test on the increase in accuracy of probability was conducted at the
.05 significance level.

First the probabilities $p_i^{irt}(\theta)$ (for original IRT) and
$p_i^{mod}(\theta)$ (for modified IRT) were calculated.  Note that each of
these formulas require $\theta$. The value for $\theta$ was calculated
per-student using the Newton-Rhapson method on the original IRT formula.  To
determine whether or not $p_i^{mod}(\theta)$ was on the whole more accurate
than $p_i^{irt}$, the result

\[
  \Delta p_{si} = (2x_{si}-1) (p_i^{mod} - p_i^{irt})
\]

was calculated. Here $p_{si}$ represents the probability that student $s$ will
answer item $i$ correctly.  Then $\Delta p_{si}$ represents the increase in
accuracy of the probability; for example, it is positive if $p_i^{mod} >
p_i^{irt}$ and $x_{si} = 1$.

Finally, to test the effectiveness of $p_i^{mod}$, a one-tailed Student's
t-test was calculated on $\Delta p_{si}$ for the dependee questions (N=648).
The test revealed that $\Delta p_{si}$ is statistically significantly greater
than zero ($t=2.05$), implying that $p_i^{mod}$ indeed produces a more accurate
estimate than IRT alone. 

\subsection{Experiment 3}

\textbf{Experiment 3 Design}.  One question remains: can $p_i^{mod}$ be used in
some meaningful way?  Suppose we wish to maximize the probability that a
student answers an Analysis question correctly, and that this Analysis question
depends on an Application question.  For that matter, it may depend on more
questions, each of which having its own degree of relatedness to the depender.

We thus posit a more general form of modified IRT which takes into account
multiple dependencies using principal axis factor (PAF) analysis
\citep{kim1978}.  PAF is a form of factor analysis which returns the proportion
of variance explained by each factor.  PAF is performed on the dependent
question using dependees as factors to obtain these proportions of variance
explained.

\begin{align*}
  p_i(\theta) &=& (1- (\displaystyle\sum_j v_j)) & \Big(\gamma_i + \frac{1-\gamma_i}{1+e^{\alpha_i(\theta-\beta_i)}}\Big) \\
              &+& \displaystyle\sum_j (v_j)   & \Big(\mathbf{sgn}(r)x_{sj}+\frac{1-\mathbf{sgn}(r)}{2}\Big)
\end{align*}

In this generalized formula, $v_j$ refers to the proportion of variance
explained by the dependee $j$ in a principal axis factor (PAF) analysis.
Therefore the sum of these is the total explained variance; one minus the sum
is the total unexplained variance.  Intuitively the probability is proportional
to the amount of variance explained by a dependee.  On the one hand, if one
dependee explains 90\% of the total variance, and the student answers this
dependee correctly, the probability should depend heavily on whether or not the
dependee was correct.  On the other hand, if the total unexplained variance is
1, this would imply the dependees are not dependees at all.  In this case, the
probability defaults to the item response theory calculation, which uses the
best available information (trait ability and question parameters).

In our experiment to test the utility of this formula, we first required a
small problem set with one depender and multiple dependees.  The experiment
was intended to test whether or not the asking of highly-dependent questions
increased mean scores.

For this we first required PAF proportions of variance between depender and
dependees.  We created 4 questions, where the 4th question had three dependees.
We then asked some of our students (N=22) to answer all of these questions to
obtain proportions of variance, which are listed below.  

\begin{center}
\begin{tabular}{|l|l|l|l|l|}
                                \hline
    &   q1 &  q2  & q3 & q4  \\ \hline
 r  &  .01 & .23 & .30 &     \\ \hline
 M  &  .67 & .67 & .44 & .33 \\ \hline
\end{tabular}
\end{center}

The others would receive a combination of questions. Four students received each
combination, such that 4 received the depender outright, 12 received one of the
three dependees (4 each) before the depender, 12 received two of the three
dependees (6 each) prior, and 4 received all three dependees.  The purpose of
this was to test the effect of exposure of each dependee.  Thus each dependee
question appeared in half of the students' assessments.  The total number of
respondents for the depender was 32, and the total number of respondents for
each dependee was 16. 

\textbf{Analysis of Experiemnt 3 Data}.  The analysis of this data required
examining the relationship between dependent response and dependee exposure.

Let $E_{sj}$ be 1 if student $s$ was exposed to $j$, 0 otherwise.  Then, our
hypothesis was that exposure of dependee questions should result in an increase
in the probability that the student answers the depender correctly, in
proportion to the degree of variance explained by the dependee.  Below is a
table giving the number of times the depender was answered correctly given
exposure to a set of dependees:

\begin{center}
\begin{tabular}{|l|l|l|}
\hline
dependees    & sum &  p   \\ \hline
none         &  1  & .25  \\ \hline
q1           &  1  & .25  \\ \hline
q2           &  2  & .50  \\ \hline
q3           &  3  & .75  \\ \hline
q1, q2       &  2  & .50  \\ \hline
q1, q3       &  2  & .50  \\ \hline
q2, q3       &  3  & .75  \\ \hline
q1, q2, q3   &  3  & .75  \\ \hline
\end{tabular}
\end{center}

Though the data set is small, there does appear to be a trend in the data. A
clearer trend can be observed by summing the number of correct responses per
exposure to a given question, which is given below:

\begin{center}
\begin{tabular}{|l|l|l|}
\hline
inclusion    & sum &  p   \\ \hline
none         &  1  & .06  \\ \hline
q1           &  8  & .50  \\ \hline
q2           & 10  & .62  \\ \hline
q3           & 11  & .69  \\ \hline
\end{tabular}
\end{center}

Due to the low $N$ in Experiment 3, the results do not present conclusive
evidence that the intervention of the dependee questions was responsible.
However, there is enough preliminary evidence to motivate further
investigation.  


\section{Conclusion}

The experiments otherwise support the hypothesis that Bloom levels do not
strictly equal difficulty levels.  There is evidence to support the
interpretation of Bloom levels and indicators of dependency relationships
between questions.  A modified form of IRT was devised, and was shown to be
effective in accounting for question dependencies.   An effort was made to
investigate the utility of this modified theory by asking intervention
questions in the form of dependees prior to a depender question.  While the
results of this latter investigation are inconclusive due to low $N$, the
preliminary evidence suggests that these interventions may have a positive
effect.  Future work will further investigate the effect of these dependency
interventions.

\bibliographystyle{apacite}
\bibliography{interact}


\end{document}
