\documentclass[a4paper,twocolumn]{article}

\setlength{\parindent}{0pt}
\setlength{\parskip}{8pt}

\title{
 The Relationship between Bloom's Cognitive Levels and Question Difficulty
}
\author{Dennis Castleberry and Steven Brandt}
\date{\today}

\begin{document}

\maketitle
\thispagestyle{empty}
\pagebreak

\abstract {

  This paper explores the relationship between Bloom's cognitive levels and
  difficulty in an intelligent tutoring system.  First, it addresses the
  question of whether or not a distinction can be made between Bloom taxonomic
  levels and question difficulty.  Second, given that a distinction can be
  made, it explores the possiblity of modelling the probability of student
  success on a given question based on its parameters and those past answered
  questions.  Third, it investigates the potential to increase that probability
  by asking "intervention questions". 

}


\section{Introduction}

\subsection{Bloom's Taxonomy}

Bloom's cognitive taxonomy organizes questions into levels depending on the
cognitive functions required of the answerer.  The levels are: knowledge,
comprehension, application, analysis, evaluation, and synthesis.  A brief
overview is given below, with definitions and examples of questions covering
the concept of for-loops:

\begin{itemize}

\item \textbf{Knowledge}. Recalling factual information.  \emph{What is a
for-loop?}

\item \textbf{Comprehension}. Assigning meaning to information.  \emph{What
does the example for-loop output? (Give example.)}

\item \textbf{Application}. Applying a rule to a specific instance.  \emph{How
can the update statement of the loop be changed to print only even numbers?}

\item \textbf{Analysis}. Breaking information into parts and exploring
relationships.  \emph{What would happen if the update statement decremented
instead of incremented the counter?}

\item \textbf{Evaluation}. Judging the use of knowledge or the validity of an
argument.  \emph{Which is better for reading user input: a for-loop or a
while-loop? Why?}

\item \textbf{Synthesis}. Utilizing knowledge to create a new solution to
satisfy a goal.  \emph{Write a for-loop to print only even numbers up to ten.}

\end{itemize}

Each category depends on the cognitive functions used in the previous category.
That is, comprehension requires knowledge, application requires comprehension,
and so forth.  Furthermore the mastery of one of the levels is with respect to
a given concept.  A student may be able to synthesize solutions to problems
dealing with expressions, but may not possess knowledge of equations, and thus
could not solve problems involving equations.

The utility of Bloom's taxonomy lies in its ability to pinpoint the underlying
cause of the student's problem-solving impasses \cite{shuhidan2011}.  Suppose a
test of mastery of loops is given with the comprehension question ``What does
such-and-such loop output?'' is given, and the student reaches an impasse.  If
the question ``What are the three expressions of the loop and what do they
do?'' is asked and the student does not know, the impasse can be attributed to
a lack of knowledge about loops.  If the student does know about the loop
expressions but still cannot answer, one might instead attribute it to a
comprehension difficulty \emph{as such}; which might be remedied by giving some
examples to build intuition, then continuing to test at the comprehension
level.

Educators may have an intuitive notion of how to do this, but Bloom's taxonomy
gives the ability to examine the impact of questions scientifically.  By
identifying the tested concept and the Bloom level of exercises, one can then
form hypotheses about student responses to questions.  


\subsection{Item Response Theory}

According to Item Response Theory (IRT), the probability $p_i$ that a
student answers correctly the $i^{th}$ question on a test, is given
by:

\[
  p_i(\theta) = \frac{1-\gamma_i}{1+e^{\alpha_i(\theta-\beta_i)}}
\]

where:

\begin{itemize} 

 \item $\alpha$ is the item discrimination, or how well the item can
 distinguish students of varying trait ability;

 \item $\beta$ is the question difficulty;

 \item $\gamma$ is the probability of guessing the answer correctly,
 which for $n$-choice questions is $1/n$;

 \item and $\theta$ is the \emph{trait ability} of the student, or the
 student's particular ability to answer that question correctly;

\end{itemize} 

An initial estimate of difficulty can be obtained from the proportion of
students with average trait ability who pass the question.


\subsection{Previous Work}

Bloom's taxonomy has been proven to be useful at the undergraduate level, and
particularly in the field of software engineering \cite{britto2015}
\cite{mahmood2014}.

EEG studies on students who solved computer science problems whose Bloom levels
were identified have shown that the levels may be distinctly clustered by EEG
signals \cite{chatterjee2015}, implying differences in brain functions used by
Bloom level.  When students are told the Bloom levels of the problems they
answer, they show a marked preference for higher-level problems
\cite{bruyn2011} \cite{goel2004}.

While it can be used to pinpoint more intellectually satisfying problems, i.e.
those at the higher Bloom levels, it has also seen success in program
comprehension \cite{buckley2003}, where the asking of comprehension questions
fosters code reading \cite{losada2008}. In addition, it has been useful for
pinpointing the difficulties of novice programmers in a guided learning
approach \cite{shuhidan2011}.

We have designed an intelligent tutoring system to explore the effect of
question ordering as it pertains to Bloom's taxonomy.  The system is based on
previous research on executable paper systems (for generating publication
manuscripts) which use R as an underlying framework \cite{castleberry2011}.
Our intelligent tutoring system is available for browsing
on-line\footnote{https://steam.cct.lsu.edu/}. It currently supports short
answer and multiple choice formats.

\section{Hypotheses}

\subsection{Preliminary Discussion}

A popular interpretation of Bloom's cognitive levels is that they are
difficulty levels, and that these difficulty levels are fixed.  Knowledge is
easy; synthesis is hard.  We will call this the Bloom-equals-difficulty
hypothesis, or $H_{B=\beta}$.

A creative interpretation of this hypothesis is that per-student difficulty can
be explained in terms of the demotion of Bloom levels.  The synthesis \emph{and
therefore difficult} questions, once the solution is obtained, are reduced to
knowledge \emph{and therefore easy} questions.  Certainly, this is one possible
scenario.  However, it does not explain a student's ability to answer
altogether distinct synthesis problems; this is to suggest students engage the
cognitive functions of synthesis, presumably as an effect of learning the use
of those functions.  To maintain $H_{B=\beta}$ would either require sustaining
the view that students are able to pass the questions because of a demotion of
the cognitive functions required (which poses a rather cynical view of
cognition and of education); or else some notion of trait ability--in
particular one which offsets difficulty, allowing an increase in the
probability of passing the question.

The latter option takes the shape of a primeval form of IRT, mapping Bloom
levels to $\beta$.  This is highly convenient for $H_{B=\beta}$, as it becomes
agreeable to evaluation with IRT.  Nontheless, there are still several issues
with $H_{B=\beta}$:

% Ideal mapping

\begin{itemize} 

\item Consider first that exposure to knowledge and knowledge questions causes
an increase in $\theta$.  According to Bloom's theory, this would increase the
probability of answering comprehension questions correctly.  This agrees with
the theory.  However, according to IRT, it would also increase the probability
of answering application questions correctly even if no comprehension-level
questions are asked.  In fact, it is theoretically possible to design tests
with sufficient numbers of knowledge questions to place $\theta-2 >> 0$.  That
is to say, asking a sufficient number of knowledge questions should give the
reasonable assurance that application-level ability is high enough not to
bother testing it.

Clearly, this is not the case. Even if a student scores exceptionally well on
knowledge questions, it gives no such assurance that the student will score
similarly on application-level questions.  Even if knowledge and application
scores are found to be correlated, there may be an additional underlying factor
contributing to both scores (such as intelligence).  The application level is
\emph{qualitatively} different.  It needs to be tested in spite of the value
of $\theta$.

% Likewise if a student is able to answer synthesis questions well, $\theta$
% should be large relative to the difficulty of knowledge questions;  $\theta -
% \beta_{knowledge} > 5$, for example.  This is an overwhelming difference
% between trait ability and difficulty.  This would imply that a student who can
% answer synthesis questions well can answer knowledge questions with high
% probability, but this is not always the case.  The student may instead be very
% creative, able to work within the confines of a limited knowledge base. 
% 
% The reader may be tempted to reject IRT for the express purpose of this
% objection; but recall that unless some notion of trait ability is introduced to
% account for variance in student responses, $H_{B=\beta}$ reduces to cognitive
% pessimism.  

\item If $H_{B=\beta}$ is strictly true, then there can exist no counterexample
to the claim.  That is to say that there exists no problem on a lower Bloom
level which is more difficult than a problem on a higher Bloom level.  To take
an example, it would be mistaken to call an application problem easy and a
comprehension problem hard. 

Counterexmaples with intuitive appeal can be constructed.  Take for-loops for
example.  To ask a student to print out the result of the loop:

\begin{center}
\begin{verbatim}
  for (int i=0; i<10; i+=2)
    printf("%d\n", i);
\end{verbatim}
\end{center}

should present an easy enough task.  This is an application of the many rules
of for-loops, but which does not require creative synthesis or critical
thinking on the part of the student, hence it is an application problem. 

% Consider then asking the student to impose a flowchart over the same code,
% including the initialization, condition and update statements and the body of
% the loop, and to indicate the start and stop of the loop.  Suppose the student
% has seen flowcharts over similar loops before.  This is a test of
% comprehensioe--in particular, comprehension of the internal workings (the
% control flow) of the loop.    
% 
% The application problem above requires only an intuitive comprehension of the
% loop: ``\emph{It prints the sequence from 0 up to 10 in steps of 2}'', the
% student may realize.  It demands only shallow comprehension and the rule to be
% applied is simple.  In the comprehension problem, on the other hand, a higher
% degree of precision in comprehension of the loop is demanded to solve the
% problem.  The comprehension problem also depends on mastery of the knowledge,
% comprehension, and application of flowchart symbols.  Regardless, explaining
% the control flow of the loop in full detail is a more difficult undertaking
% than simply printing out its output. 

Likewise, it is possible to ask the student to synthesize a loop printing the
first ten powers of two, then ask a comparatively difficult analysis question
in the form of an obfuscated loop code, such as asking for the value of
\texttt{s} for each iteration of the following loop: 

\begin{verbatim}
    int i, s=1001;
    for (i=10; i>0; i--) 
      s %= (int) pow(2.0, i);
\end{verbatim}

Here, the impasse may be attributed to lesser-known symbols such as
\texttt{\%=}, backwards looping, typecasting, and otherwise unusual behavior.
In the synthesis problem, the student has the choice to work with familiar
constructs.

\item By Bloom's own admission, it is possible to skip levels for certain
concepts.  In this case, the ordinality of Bloom levels breaks down.  A
plausible example is given above.  It is possible to teach how to obtain the
numeric sequences printed by for-loops before covering in full detail the
control flow of the loop.   

\end{itemize}

With this in mind, we would like to introduce $H_{B\neq\beta}$: simply put,
Bloom levels do not equate to difficulty.  Rather, Bloom levels indicate
\emph{dependency relationships}.  As the theory goes, knowledge is a
\emph{requirement} for comprehension; but this is not to say that the
acquisition of the knowledge is somehow easier.  

To evaluate $H_{B\neq\beta}$, we introduce the notion of an intelligent
tutoring system.  An intelligent tutoring system is a computer-based system
which interacts with a student with the end-goal of teaching material, but
adapts its interaction in some way to the student's responses, much as a human
tutor would.  With such a system, we can automate the scheduling of questions,
collect responses, and analyze them. For the purpose of evaluating
$H_{B\neq\beta}$, we can associate difficulty and Bloom levels with questions,
then collect responses to test the hypothesis statistically.

\subsection{Research Questions}

\textbf{Question 1}. Is there a difference between Bloom level and difficulty?
(That is, can an appreciable amount of variance in student responses be
attributed to Bloom level?)

\textbf{Question 2}. Given that there exists a difference between Bloom level
and difficulty, is it possible to modify IRT to more accurately predict
$p(x_i)$ given Bloom level information?

\textbf{Question 3}. Given a predictive model and parameters for a question
$x_i$, is it possible to manipulate a target probability $p(x_i)$ by asking
related questions?

\section{Methodology and Analysis}

Our intelligent tutoring system has a bank of questions which are tagged by
Bloom level, concept, and question difficulty.  We can use our system to easily
create groups of questions (which we call assessments) and issue them either on
paper or electronically.  Our intelligent tutoring system supports short-answer
and multiple-choice formats, and can provide immediate feedback to the student
on the correctness of their response.

\textbf{Experiment 1}.

\textbf{Experiment 2}.

\textbf{Experiment 3}.

\section{Analysis and Results}

\textbf{Experiment 1}.

\textbf{Experiment 2}.

\textbf{Experiment 3}.

\section{Limitations}


\section{Conclusion}


\bibliographystyle{plain}
\bibliography{main}

\end{document}
