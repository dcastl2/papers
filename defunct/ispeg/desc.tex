\section{Problem Definition}

Consider the \textbf{pattern} $(aa|a)a$ and the \textbf{string} ``aa''.
Then $(aa|a)$ is a \textbf{subpattern} of the \textbf{total pattern}
$(aa|a)a$.  Let the choice $(aa|a)$ be called an \textbf{ordered choice}
%\todo{SRB: The official terminology for what you are calling a disjunction is an ``ordered choice.''}
(OC) of the pattern $(aa|a)a$. Let $aa$ and $a$ be called the
\textbf{choices} of the subpattern $(aa|a)$.  Define three
types of parsing expression grammars (PEGs) based upon matching behavior:

\begin{itemize}

  \item \textbf{O-PEG}. Given any OC in a pattern, an ordinary
    (\emph{O-PEG}) consumes the substring whose first OC in the
    subpattern matches.  Given the above example, the OC $aa$
    will be matched against the string ``aa''. Once matched, the string ``aa''
    is consumed and all other OCs in the subpattern are ignored. When
    O-PEG encounters the subpattern $a$ in the remainder of the total
    pattern, it fails to match the now-empty string ``''; thus O-PEG fails to
    match the string against the total pattern.

  \item \textbf{CS-PEG}. The current streamlined PEG (or \emph{CS-PEG}) attempts
    to match patterns of a subpattern simultaneously.  An S-PEG consumes the
    first substring whose OC matches in the context of the total pattern.
    It does so by splitting the choice pattern into two or more
    non-choice patterns, then attempting to match them independently.
    These attempts are embodied by \textbf{matching streams}. When any matching
    stream succeeds in matching an OC, it kills all other 
    % \todo{srb: only child matching streams} 
    child matching streams
    since they should be unnecessary. However, the following example
    illustrates why this behavior is flawed.
    
    In the above example, the pattern $(aa|a)a$ is broken into two
    patterns $(aa)a$ and $(a)a$, and two simultaneous matching
    streams $M_1$ and $M_2$ are spawned which attempt to match against the
    string simultaneously and character-by-character.  Both patterns consume
    the first character `a' in the string, leaving the substring ``a''.  When
    subpattern $(aa)$ of $M_1$ consumes the second character and reaches
    the end of the OC, it kills $M_2$, because $M_1$ has succeeded up to
    this point, indicating that other matching streams are unnecessary. The
    remaining substring is ``'', which $M_1$ fails to match against the
    subpattern $a$. Thus $M_1$ reports failure, and since $M_2$ was
    killed, both $M_1$ and $M_2$ report failure and the CS-PEG fails to match
    the string against the total pattern.

  \item \textbf{IS-PEG}. The ideal streamlined PEG (or \emph{IS-PEG}) attempts
    to match subpatterns of an OC simultaneously, similarly to an CS-PEG.
    However, it does not succumb to the above-illustrated error which the CS-PEG
    does. Given the above example, the difference in IS-PEG's behavior occurs
    when $M_2$ matches the subpattern $(a)$; upon doing so, it flags its
    success up to this point.  Those both $M_1$ and $M_2$ consume the first
    ``a''. When $M_1$ consumes the second `a' in its OC, it then attempts
    to kill all other streams. It checks $M_2$'s success flag, and does not
    kill $M_2$ because its flag is set. $M_2$ then consumes the final `a',
    reaches the end of the string, and reports success. Thus IS-PEG matches
    the string against the total pattern.

\end{itemize}

