\section{Example Cases}

\subsection{A Streaming PEG and RHS-Disjuncts}

Consider the pattern $(aa|a)a$ with the text ``aa''. While a traditional PEG
fails on this, a streaming PEG can succeed. The pattern is broken by its
disjunct into two subpatterns $(aa)a$ and $(a)a$ which are handled by two
matcher streams $M_1$ and $M_2$, respectively. 

\begin{figure}[h]
\begin{center}
\begin{tabular}{lccc}
Iteration & $M_1$ Pattern & $M_2$ Pattern & Input  \\
0         & $(aa)a$       & $(a)a$        & ``aa'' \\
1         & $(a)a$        & $a$           & ``a''  \\
2         & $a$           &               & ``''   \\
3         & \Cross        & \Check        & ``''   \\
\end{tabular}
\end{center}
\label{tab:case1}
\caption{Pattern $(aa|a)a$ with input ``aa''.}
\end{figure}

Further in Iteration 2, the $M_1$ pattern is $a$ while the input is ``''.
Since $M_1$ finds that it has characters to be consumed in its remaining
subpattern although there is no remaining input, it sends a fail signal to be
propagated during the next iteration.  Also at Iteration 2, $M_2$ discovers
that both its subpattern and its input are empty, which prompts it to send a
success signal to be propagated during the next iteration. Thus the pattern
matches.


\subsection{IS-PEG vs. CS-PEG and the Partial Success Flag}

This example illustrates how the partial success flag influences what disjuncts
are killed.  Now consider the pattern $(aa|a)aa$ with the text ``aaa''.  As
before, the pattern is broken by its disjunct into two subpatterns $(aa)aa$ and
$(a)aa$ handled by $M_1$ and $M_2$. 

\begin{figure}[h]
\begin{center}
\begin{tabular}{lccc}
Iteration & $M_1$ Pattern & $M_2$ Pattern & Input  \\
0         & $(aa)aa$      & $(a)aa$       & ``aaa''\\
1         & $(a)aa$       & $aa$          & ``aa'' \\
2         & $aa$          & $a$           & ``a''  \\
3         & $a$           &               & ``''   \\
4         & \Cross        & \Check        & ``''   \\
\end{tabular}
\label{tab:case2}
\caption{Pattern $(aa|a)aa$ with input ``aaa''.}
\end{center}
\end{figure}

The characters within each pattern are consumed as they are matched to
characters in the input string, up to Iteration 3.  By the beginning of
Iteration 3,  CS-PEG  exits the disjunct $(aa)$ and, since it has succeeded up
to this point, sends a kill signal to $M_2$, which dies before it can report
its success in Iteration 4. 

In  IS-PEG , by Iteration 3 $M_2$ has its partial success flag set to indicate
that it has succeeded up until this point; thus indicating that it is possible
that $M_2$ could succeed in its match where $M_1$ may fail. Though $M_1$ does
send a kill signal, $M_2$ does not die.  The process beyond Iteration 3 follows
the same course as the previous example; thus the match succeeds with $M_2$.


\subsection{Preventing Unnecessary Streams}

Now consider the rule $A := (aA|)$ with the text ``aaaaa''. In this example,
matcher streams are spawned for each recursion into the rule $A$. They are
unnecessary, yet they terminate quickly; thus the problem in such a case is
benign. However, in the rule $A := (aA|)(aA|)(aA|)$ with text ``aaaaa'', an
exponential number of streams is spawned with each recursion into the rules
$A$.  In the example, it would require 2 recursive calls for one matcher stream
to consume the total input--by which point $1+3+9$ for the rule $A$ and $9$ for
the empty rule, for a total of 22 (not counting signal streams)--will exist.
These streams must be dealt with early to ensure the complexity of parsing does
not balloon. 

\begin{figure}[h]
\begin{center}
\begin{BVerbatim}
  0        1        2 ...

(aA|)
  |
  |------(aA|)
  |        |
  |        |------(aA|)
  |        |------(aA|) *
  |        \------(aA|)
  |
  |------(aA|)
  |        |
  |        |------(aA|)
  |        |------(aA|)
  |        \------(aA|)
  |
  \------(aA|)
           |
           |------(aA|)
           |------(aA|)
           \------(aA|)
\end{BVerbatim}
\end{center}
\caption{The BFS-like expansion of $A := (aA|)(aA|)(aA|)$. The star 
(*) marks the stream at which the pattern matches.}
\end{figure}

% The input will be parsed as (a)(a(a)(a))(a)(a).

The reason for the explosion is due to the nature of the stream expansion: it
is done in a fashion similar to breadth-first search (BFS).  For this example,
imagine the rule expansion as a ``node''.  A BFS-like expansion will guarantee
an exponential number of matcher streams due to the three-fold rule expansion;
yet it will succeed in two iterations.  Now consider a DFS-like expansion.

\begin{figure}[h]
\begin{center}
\begin{BVerbatim}
  0        1        2        3        4

(aA|)
  |
  |------(aA|)
  |        |
  |        |------(aA|)
  .        |        |------(aA|)
  .        .        |        |------(aA|) *
  .        .        .        .
\end{BVerbatim}
\end{center}
\caption{The DFS-like expansion of $A := (aA|)(aA|)(aA|)$. The star 
(*) marks the stream at which the pattern matches.}
\end{figure}

In this expansion, 5 iterations are taken but result in the use of only 5
matcher streams. 

\todo{SRB: So the question is, what is to be done? I think the semantics of the S-PEG must
match those of the O-PEG. What do you think?}

% TODO

\pagebreak

Consider $s = ``aabb''$ and $r = (a|ab|aab)b$.

\begin{figure}[h]
\begin{center}
\begin{BVerbatim}
      0       1        2       3       4

A     |------abb--||---x----------------------
      |
B   aabb-----abb-------x---||-----------------
      |
C     |------abb------abb------bb--||--b------
\end{BVerbatim}
\end{center}
\end{figure}

\texttt{Match-All} should consume at most one character per iteration.  Given
an OC rule, the end-of-rule or kill marker (\texttt{||}) indicates that this stream
should kill all other streams if its partial success flag is true.  In the
above example, stream $A$ hits its kill marker at the end of iteration 1, and
its partial success flag is true. Thus it should kill $B$ and $C$.  However
killing $C$ is premature, since on the next iteration $A$ will fail, and $C$
would succeed if it were not killed.
