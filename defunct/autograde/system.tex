\subsection{System Design}

The directory hierarchy within the VM uses lexicographically-ordered assignment
and student IDs to support batch operations. Fig.~\ref{fig:layout} gives an example of
a typical layout.

\begin{figure}
  \begin{Verbatim}[frame=single]
    prog1/
        user135001/
            Hello.java
            Arithmetic.java
        user135002/
            Hello.java
            Arithmetic.java
        ...
    prog2/
        user135001/
            Shopping.java
            Rocket.java
        ...
    ...
\end{Verbatim}
  \caption{The directory hierarchy lends itself to straightforward batch processing.}
  \label{fig:layout}
\end{figure}

Generally, the ID is assigned as $prefix\_m\_n$, where $m$ is the course number
and $n \in N$ is a student ID.  The value $n$ is mapped to a name $name$, as in
$n \rightarrow name$. The string $n$ is constructed such that for all $n_1, n_2
\in N$ and all $name_1, name_2 \in Names$, if $n_1 \rightarrow name_1$, $n_2
\rightarrow name_2$ and if $name_1 <_{lex} name_2$, then $n_1 <_{lex} n_2$.
That is, the IDs are constructed to be in lexicographic order by student last
name.  Since students may add the course late, we make the ID distribution
sparse to assign such students an appropriate position ID in the lexicographic
ordering.  
%\todo{ Will humans be looking at these id's? If so, using the students name might be preferrable to a number.}

The Autograde script handles steps 1-4 of \ref{sec:workflow}. The auto-evaluation
is handled by Java programs. 

