\subsection{Automatically Generating a Grammar}

While a grammar and grading scheme are sufficient to automatically grade a
document set, writing an Autograde grammar can be as tedious and time-consuming
as manually grading the document set.  When a document set has multiple
possible correct solutions which vary in the slightest--such as an added space,
brace, or parenthesis--a strict grammar can cause a match to fail.  Choice
operators help to avoid this, but can be painstaking to tune.  Adding them
requires that the user scan over the document set to tell where the text should
be interpreted as correct, which potentially defeats the point of using an
automated grading system.

To remedy this, we have created an algorithm to generate a line-by-line grammar
rule set from a collection of documents which are known to be correct.  The
Algorithm \texttt{Generate-Grammar} is depicted in Fig~\ref{alg:gengrammar}.
The algorithm initializes its rule set to the lines within the first document
in the list.  For each document thereafter, it attempts to match each rule
against the text; if it succeeds nothing happens, but if it fails, it stores
the index within the text where it fails in $n$.  Both the current rule and the
line of text are then reversed, then a match is attempted.  It cannot succeed;
however it can match up to at least $n$.  At most it will match up to $m$; that
is however far the backwards match succeeds. The array of characters from $n$
to $m$ form a bounds on the part of the text which fails to match; this array
of characters is thus added as an ordered choice within the grammar rule.

\begin{figure}[tr]
  \begin{framed}
Data Structures:
\begin{itemize}
  \item A document array $\textbf{D}$ containing documents known to be fully
  correct, wherein each $d_i \in D$ there exists a set of lines $L_i$, where
  $l_{ij}$  is the $j^{th}$ line of the $i^{th}$ document.
  \item A rule set $R$; initially let $R \leftarrow \empty$.
\end{itemize}

Algorithm $\textbf{\texttt{Generate-Grammar}}(D)$:
\begin{enumerate}
  \item For all $d_i \in D$:
  \begin{enumerate}
    \item If $i == 0$, let $r_j \leftarrow l_{0j}$.
    \item Else: 
      \begin{enumerate}
	\item Let $n \leftarrow \texttt{match}(r_j, d_i)$.
	\item If $n \neq -1$:
	  \begin{enumerate}
	    \item Let $r_j' \leftarrow \texttt{Reverse-Rule}(r)$.
	    \item Let $m    \leftarrow \texttt{Parse}(r_j', d_i)$.
	    \item Let $r_j  \leftarrow \texttt{Add-Choice}(r_j, l_{ij}, n, m)$.
	  \end{enumerate}
      \end{enumerate}
  \end{enumerate}
  \item Return $R$.
\end{enumerate}
\end{framed}
\label{alg:gengrammar}
\caption{The \texttt{Generate-Grammar} algorithm.}
\end{figure}

To reverse a rule, we swap all children of an ordered choice (Or) operator, and
reverse all items in sequence (Seq).  Details of this operation are shown in
Fig~\ref{alg:revgrammar}.

\begin{figure}[tr]
\begin{framed}
  Data Structures:
  \begin{itemize}
    \item A rule $r$ may be defined recursively as:
  \end{itemize}
    \begin{verbatim}
      Rule r = Lit  x
             | Seq  x y
             | Or   x y
             | Star x
             | Plus x
             | Nil   
    \end{verbatim}

Algorithm $\textbf{\texttt{Reverse-Rule}}(r)$:
\begin{enumerate}
  \item Construct the parse tree $T$ for $r$.
  \item For each node $n_i$ in an in-order traversal of $T$:
    \begin{enumerate}
      \item If $n_i$ is of type \texttt{Seq}:
	\begin{enumerate}
          \item Append $(i, n_i)$ to $S$.
	\end{enumerate}
      \item Else if $n_i$ is of type \texttt{Or}:
	\begin{enumerate}
          \item Swap the children of $n_i$.
	 \end{enumerate}
    \end{enumerate}
  \item Let $S  \leftarrow \texttt{Reverse}(S)$.
  \item Let $T' \leftarrow \texttt{Merge}(S, T)$;
\end{enumerate}
\end{framed}
\label{alg:revgrammar}
\caption{The \texttt{Reverse-Rule} algorithm.}
\end{figure}
