\subsection{Auto-Evaluation}

% TODO: add citation for Piraha.
Automated evaluation of source code, output, and error messages is done by
parsing using a parsing expression grammar (PEG).  The PEG of choice is Piraha,
a simplistic PEG which supports dynamic compilation of grammar rules.
Auto-evaluation works by parsing grading scheme rules which tell (a) what
expressions to parse, (b) how many points to award if the given expressions are
successfully parsed and (c) error annotations (comments) which are given to the
students if either positive-point expressions are not correctly parsed or
negative-point expressions are.  

The following is an example of a grading scheme file.  In this file, 5 points
are awarded for the correct parsing of rules \texttt{ins-array} and
\texttt{sel-array}.  If they are not parsed, the line ``ins. sort incorrect''
or ``sel. sort incorrect'' is printed:

\vspace{4pt}
\small
  \begin{Verbatim}
  5:  ins-array  "ins. sort incorrect"
  5:  sel-array  "sel. sort incorrect"
  \end{Verbatim}
\normalsize
\vspace{4pt}

The rule names are defined in a separate grammar file which tells how the text
in the document should be parsed. In the following example, the rules are equal
to the string output of array contents at certain steps in the execution:

\vspace{4pt}
\small
\begin{Verbatim}
  ins-array = [3, 5, 4, 6, 8, 7]
  sel-array = [3, 4, 5, 6, 8, 7]
\end{Verbatim}
\normalsize
\vspace{4pt}

A more complex grammar scheme file may involve constructs in the source code
which need to be parsed, in which case the user may which to deduct points
for poor practices or transgressions of style:

\vspace{4pt}
\footnotesize
\begin{Verbatim}
   2: ins-sort-sig       "wrong sig. for ins. sort"
      .
      .
      .
  -1: no-space-after  "no space after if/for/while"
\end{Verbatim}
\normalsize
\vspace{4pt}

In the following grammar, the group \texttt{\{psv\}}, elsewhere defined, refers
to the string ``public static void''. The method name is stipulated to be
\texttt{insertionSort} and the sole argument may be any alphanumeric name
covered by the rule \texttt{\{alpha\}}, elsewhere defined. The group
\texttt{\{no-space-after\}} refers to an if-, for-, or while- construct keyword
followed by no whitespace. 

\vspace{4pt}
\scriptsize
  \begin{Verbatim}
  ins-sort-sig      =  {psv} insertionSort (int[] {alpha})
    .
    .
    .
  no-space-after    =  (if|for|while)\(
\end{Verbatim}
\normalsize
\vspace{4pt}

Algorithm \texttt{Auto-Eval} is depicted in Fig~\ref{alg:autoeval}.  For every
item in the grading scheme, matching is attempted on the corresponding grammar
rule.  If it succeeds where the mark is positive, the mark is added to the
score total; where the mark is negative, the negative mark is deducted from the
score total and a comment is given.  If matching fails on a positive-mark item,
a comment is given; and if it fails on a negative-mark item, nothing happens.
This agrees with our intuitive notion of grading: we should award points for
correct items, fail to award points for incorrect, deduct points for
transgressions that do occur, and fail to deduct when they do not; comments
should be given whenever points are deducted from or not added to the sum.


\begin{figure}[tr]
\begin{framed}
Data Structures:
\begin{itemize}
  \item A rule set $R$ for parsing where each element of $R$ is a tuple $(e_r, r)$.
  \item A document $d$ to match rules against.
  \item A grading scheme rule set $S$ where each element of $S$ is a tuple $(g, e_s, c)$, where:
    \begin{itemize}
      \item $g$ is the grade, and
      \item $e_r$ is an expression such that $e_r \in R$.
    \end{itemize}
\end{itemize}

Algorithm \texttt{Auto-Eval($d, R, S$)}:
\begin{enumerate}
 \item For each item $(g, e_s, c) \in S$:
  \begin{enumerate}
    \item Let $success = match(e_s, d)$.
    \item If ($success == true$):
    \begin{enumerate}
      \item If ($g > 0$): $sum = sum + g$
      \item Else: 
	\begin{enumerate}
	  \item $sum = sum + g$ 
	  \item $stdout \leftarrow c$
	\end{enumerate}
    \end{enumerate}
    \item Else:
    \begin{enumerate}
      \item If ($g > 0$): $stdout \leftarrow c$
      \item Else:         \emph{pass}
    \end{enumerate}
  \end{enumerate}
  \item Return $sum$.
\end{enumerate}
\end{framed}
\caption{The Auto-Eval algorithm.}
\label{alg:autoeval}
\end{figure}

