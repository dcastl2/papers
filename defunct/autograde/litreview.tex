\section{Literature Review}

%The concept of an automated grading system is by no means new.
As early as 1973, automated grading had been developed for COBOL programs
\cite{aaronson}.  However, like many automated systems which exist today, it
only compared the output of the student code and solution code \cite{pieterse,
gotel, singh, nordquist, edwards, edwards2, sherman}.  Systems with this
limitation have the advantage of being modularly language-independent,
allowing support for new languages to be added as required. Aside from
multi-language generalization, the auto-grading model itself has not seen
much change since 1973.

The all-or-nothing character of output auto-grading systems, \cite{zanden}, in
which the total assignment or sub-assignment is correct if and only if the
output exactly matches that of the solution code, leaves something to be
desired.  With such systems, a single ``off-by-one'' index error potentially
yields a grade of zero.  One particular needs-specific system, implemented for
Java, uses reflection to test the correctness of methods \cite{helmick}. The
Java Reflect API allows for methods from a student and solution code to be
invoked on user-controlled inputs and their return values, fields, etc. to be
compared to determine if the methods are algorithmically equivalent in spite of
semantic differences in the student and solution main methods. This approach is
desirable for more refined control of the grading.

A related goal of automated grading is to provide expeditious feedback both on
laboratory exercises \cite{nordquist} and programming assignments \cite{singh},
for which an auto-suggestion module would ideally be integrated into the
auto-grading system and trigger in the event of an error.

% The implications of auto-grading \cite{holton, sherman}.

A web-based assignment submission and assessment interface is a worthy
component because of the system's ability to be used remotely and
platform-independently via controls which are intuitive to students \cite{fu,
gotel, edwards2}.

% Caution that not everything should be automated \cite{harris}.
