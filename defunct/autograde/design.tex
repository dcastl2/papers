\subsection{Design Logic}


We propose the following features for an autograding system:

\begin{itemize}

  \item \textbf{Duplicable and deployable system}. We should be able to copy
    the total system easily onto another machine and expect it to work without
    having to install any additional software, configure anything, or otherwise
    re-trace the author's steps.

  \item \textbf{Non-interactive batch processing}. Because programs can number
    in the hundreds, ideally the user should be able to tell the auto-grader
    everything to do prior to a run. Should errors with a student code occur,
    they should not hamper the processing of others.
    
  \item \textbf{Minimum description length}. The user should be able to program
    an autograde run by writing (a) what to check and (b) how to respond if the
    check passes or fails. The rest should be handled automatically.

  \item \textbf{Language modularity}. It should not simply work with popular
    languages; it should be made to work for any.  If a new language is
    developed, a user should be able to write a plug-in for it rather than
    wait for hard-coded support.

  \item \textbf{Processing modularity}. The steps within the auto-grading
    process should themselves be modules to fully leverage language modularity
    and simplify the process of modifying the code base.

  \item \textbf{Student profiling}. Given the available information on
    compilation and execution errors, an agent should be able to report
    on a student's strengths and weaknesses, in turn informing future
    assignment specifications.

\end{itemize}

To satisfy the need for duplicability and deployability, the current Autograde
implementation exists within a virtual machine (VM) running LAMP.  The total
VM may be copied and run in any operating system.

Because batch processing of files is a pivotal task, the interface to the
system as well as its modules are written in Linux shell.  This allows for ease
of editing modules, as new programs which are developed may be incorporated.

The minimum description length for a grading specification is achieved by two
user-supplied files: a file for parsing, and a grading specification file for
point values based upon parsed expressions.


