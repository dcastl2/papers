% A review of computer-assisted assessment, 
@article{conole2005review,
  title={A review of computer-assisted assessment},
  author={Conole, Gr{\'a}inne and Warburton, Bill},
  journal={ALT-J},
  volume={13},
  number={1},
  pages={17--31},
  year={2005},
  publisher={Taylor \& Francis}
}

% Concerned with the creation of objective questions at the higher Bloom levels.
% Outlines a method for doing so.  Shows that objective questions at the higher
% levels can be created.
@article{duke2001using,
  title={Using computer-aided assessment to test higher level learning outcomes},
  author={Duke-Williams, Emma and King, Terry},
  year={2001},
  publisher={{\copyright} Loughborough University}
}

% Stresses the need for a competency model, namely one that uses Bloom's taxonomy of
% educational objectives.  Suggests that information in a competency model can be
% used to personalize assessment.
@article{sitthisak2007towards,
  title={Towards a competency model for adaptive assessment to support lifelong learning},
  author={Sitthisak, Onjira and Gilbert, Lester and Davis, Hugh C},
  journal={Service Oriented Approaches and Lifelong Competence Development Infrastructures},
  pages={200},
  year={2007},
  publisher={Citeseer}
}

% In biology, it is agreed that there is a matter of correctness about the answers
% to higher-order questions.
@article{lemons2013questions,
  title={Questions for Assessing Higher-Order Cognitive Skills: It's Not Just Bloom’s},
  author={Lemons, Paula P and Lemons, J Derrick},
  journal={CBE-Life Sciences Education},
  volume={12},
  number={1},
  pages={47--58},
  year={2013},
  publisher={Am Soc Cell Biol}
}

% The CAT prototype introduced here includes a proficiency level estimation
% based on Item Response Theory and a questions’ database. The questions in the
% database are classified according to topic area and difficulty level. The level
% of difficulty estimate comprises expert evaluation based upon Bloom’s taxonomy
% and users’ performance over time.
% Apparently the difficulty is set to the Bloom level. No scheduler.
@article{lilley2005generation,
  title={The generation of automated learner feedback based on individual proficiency levels},
  author={Lilley, Mariana and Barker, Trevor and Britton, Carol},
  journal={Innovations in Applied Artificial Intelligence},
  pages={14--15},
  year={2005},
  publisher={Springer}
}

% Mentioned IRT and bloom, but it is not clear how these are combined or how
% they are scheduled. It seems that the assessments themselves are scheduled. 
% This is a prototype system.
@article{yarandi2011personalised,
  title={Personalised mobile learning system based on item response theory},
  author={Yarandi, Maryam and Jahankhani, Hamid and Dastbaz, Mohammad and Tawil, Abdel-Rahman},
  year={2011}
}

% examines several multiple choice formats 
@article{haladyna1992effectiveness,
  title={The effectiveness of several multiple-choice formats},
  author={Haladyna, Thomas M},
  journal={Applied Measurement in Education},
  volume={5},
  number={1},
  pages={73--88},
  year={1992},
  publisher={Taylor \& Francis}
}

% empirical support for formative assessment, feedback-based learning
@article{lawton2012online,
  title={Online learning based on essential concepts and formative assessment},
  author={Lawton, Daryl and Vye, Nancy and Bransford, John and Sanders, Elizabeth and Richey, Michael and French, David and Stephens, Rick},
  journal={Journal of Engineering Education},
  volume={101},
  number={2},
  pages={244--287},
  year={2012},
  publisher={Wiley Online Library}
}

% ----------------------------------------------------------------------------------

% Duplicate of my research, more importantly suggests outright that Bloom=difficulty
@article{newman1988effect,
  title={Effect of varying item order on multiple-choice test scores: Importance of statistical and cognitive difficulty},
  author={Newman, Dianna L and Kundert, Deborah K and Lane Jr, David S and Bull, Kay Sather},
  journal={Applied Measurement in education},
  volume={1},
  number={1},
  pages={89--97},
  year={1988},
  publisher={Taylor \& Francis}
}

% Very clear support for the thought that Bloom=difficulty, even used for course difficulty rating
@inproceedings{oliver2004course,
  title={This course has a Bloom Rating of 3.9},
  author={Oliver, Dave and Dobele, Tony and Greber, Myles and Roberts, Tim},
  booktitle={Proceedings of the Sixth Australasian Conference on Computing Education-Volume 30},
  pages={227--231},
  year={2004},
  organization={Australian Computer Society, Inc.}
}

% Very clear support for the thought that Bloom=difficulty
@article{lord2007moving,
  title={Moving students from information recitation to information understanding: exploiting Bloom's taxonomy in creating science questions},
  author={Lord, Thomas and Baviskar, Sandhya},
  journal={Journal of College Science Teaching},
  volume={36},
  number={5},
  pages={40},
  year={2007},
  publisher={National Science Teachers Association}
}

% Support for people thinking that Bloom=difficulty
@inproceedings{johnson2006bloom,
  title={Is Bloom's taxonomy appropriate for computer science?},
  author={Johnson, Colin G and Fuller, Ursula},
  booktitle={Proceedings of the 6th Baltic Sea conference on Computing education research: Koli Calling 2006},
  pages={120--123},
  year={2006},
  organization={ACM}
}

% Support for people thinking that Bloom=difficulty
@inproceedings{fuller2007developing,
  title={Developing a computer science-specific learning taxonomy},
  author={Fuller, Ursula and Johnson, Colin G and Ahoniemi, Tuukka and Cukierman, Diana and Hern{\'a}n-Losada, Isidoro and Jackova, Jana and Lahtinen, Essi and Lewis, Tracy L and Thompson, Donna McGee and Riedesel, Charles et al.},
  booktitle={ACM SIGCSE Bulletin},
  volume={39:4},
  pages={152--170},
  year={2007},
  organization={ACM}
}


% Support for Bloom!=difficulty: facility versus complexity
@article{hill1981testing,
  title={Testing the simplex assumption underlying Bloom's Taxonomy},
  author={Hill, PW and McGaw, B},
  journal={American Educational Research Journal},
  volume={18},
  number={1},
  pages={93--101},
  year={1981},
  publisher={Sage Publications}
}

% Support for Bloom!=difficulty
@inproceedings{thompson2008bloom,
  title={Bloom's taxonomy for CS assessment},
  author={Thompson, Errol and Luxton-Reilly, Andrew and Whalley, Jacqueline L and Hu, Minjie and Robbins, Phil},
  booktitle={Proceedings of the tenth conference on Australasian computing education-Volume 78},
  pages={155--161},
  year={2008},
  organization={Australian Computer Society, Inc.}
}


% Cursory
@article{sitthisak,
  title={Cognitive Assessment Applying with Item Response Theory},
  author={Sitthisak, Onjira and Soonklang, Tasanawan and Gilbert, Lester},
  journal={19th Annual International Conference on Computers in Education},
  year={2011}
}

% Bloom original
@book{bloom1956,
  title={Taxonomy of educational objectives},
  author={Bloom, Benjamin Samuel et al.},
  year={1956},
  publisher={David McKay}
}

% Factor analysis
@book{kim1978,
  title={Factor analysis: Statistical methods and practical issues},
  author={Kim, Jae-On and Mueller, Charles W},
  volume={14},
  year={1978},
  publisher={Sage}
}

% Newton Raphson
@book{baker2004,
  title={Item response theory: Parameter estimation techniques},
  author={Baker, Frank B and Kim, Seock-Ho},
  year={2004},
  publisher={CRC Press}
}

% R can be hooked to paper manuscripts to re-generate scientific findings.
@article{ castleberry2011,
          author    = {Dennis Castleberry}
         ,title     = {The Prickly Pear Archive}
         ,journal   = {ICCS}
         ,address   = {Singapore}
         ,publisher = {Elsevier}
         ,year      = {2011}
}

% ...26 studies were deemed as relevant. The main findings from these studies
% are: i) Bloom’s taxonomy has mostly been applied at undergraduate level for
% both design and assessment of software engineering courses; ii) software
% construction is the leading SE subarea in which Bloom’s taxonomy has been
% applied. The results clearly point out the usefulness of Bloom’s taxonomy in
% the SE education context.
@article{ britto2015,
   author = {Ricardo Britto and Muhammad Usman}
  ,title  = {Bloom’s Taxonomy in Software Engineering Education: A Systematic Mapping Study}
  ,journal= {Frontiers in Education Conference}
  ,address= {El Paso, TX}
  ,publisher= {IEEE}
  ,year   = {2015}
}

% Feedback from the students from both first year and second year reflected a
% positive response concerning their perception of the level on which the
% questions were asked. Regarding the first question, 381 students felt that by
% asking questions on higher levels of Bloom’s taxonomy it enriched their
% learning
@article{ bruyn2011
  ,author = {E de Bruyn and E Mostert and A van Schoor}
  ,title  = {Computer-based testing--the ideal tool to assess on the different levels of Bloom’s taxonomy}
  ,journal= {Interactive Collaborative Learning}
  ,address= {Piesany, Slovakia}
  ,publisher= {IEEE}
  ,year   = {2011}
}

% Bloom's taxonomy as applied to program comprehension was successful
@article{ buckley2003
  ,author = {Jim Buckley and Chris Exton}
  ,title  = {Blooms’ Taxonomy: A Framework for Assessing Programmers’ Knowledge of Software Systems}
  ,journal= {International Workshop on Program Comprehension}
  ,address= {}
  ,publisher= {IEEE}
  ,year   = {2003}
}

% Clustering of tasks using EEG is possible
@article{ chatterjee2015,
   author = {Debatri Chatterjee and rajat Das and Anirhudda Sinha and Shreyasi Datta}
  ,title  = {Analyzing Elementary Cognitive Tasks with Bloom's Taxonomy using Low Cost Commercial EEG device} 
  ,journal= {International Conference on Intelligent Sensors, Sensor Networks and Information Processing (ISSNIP)}
  ,address= {}
  ,publisher= {IEEE}
  ,year   = {2015}
}

% Preference for higher-order cognitive activities
@article{ goel2004,
   author = {Sanjay Goel and Nalin Sharda}
  ,title  = {What do engineers want? Examining engineering education through Bloom's taxonomy}
  ,journal= {Australasian Association for Engineering Education}
  ,address= {}
  ,publisher= {}
  ,year   = {2004}
}

% Context-aware vs. lexical schema
% Successful application to system
@article{ kelly2006,
   author = {Tara Kelly and Jim Buckley}
  ,title  = {A Context-Aware Analysis Scheme for Bloom's Taxonomy}
  ,journal= {International Conference on Program Comprehension}
  ,address= {}
  ,publisher= {IEEE}
  ,year   = {2006}
}

% several points aobut requirements
%  formal language
%  typing system
%  Bloom's semantics
@article{ loria-saenz2008,
   author = {Carlos Loria-Saenz}
  ,title  = {On Requirements for Programming Exercises from an e-Learning Perspective}
  ,journal= {}
  ,address= {}
  ,publisher= {}
  ,year   = {2008}
}

% increase in motivation
% comprehension testing fosters code reading
@article{ losada2008
  ,author = {Isidoro Hern\'{a}n-Losada}
  ,title  = {Testing-Based Automatic Grading: A Proposal from Bloom’s Taxonomy}
  ,journal= {International Conference on Advanced Learning Technologies}
  ,address= {}
  ,publisher= {IEEE}
  ,year   = {2008}
}

% speaks well for bloom's taxonomy when manually implemented
@article{ mahmood2014
  ,author = {Mahmood Niazi}
  ,title  = {Teaching global software engineering: experiences and lessons learned}
  ,journal= {IET Software}
  ,address= {}
  ,publisher= {}
  ,year   = {2014}
}

% speaks well for use of Bloom's taxonomy in software inspection
@article{ mcmeekin2009
  ,author = {David McMeekin and Brian von Konsky and Elizabeth Chang and David Cooper}
  ,title  = { }
  ,journal= { Conference on Software Engineering Education and Training }
  ,address= {}
  ,publisher= {IEEE}
  ,year   = {2009}
}

% good for novice programmers to dial it down
@article{ shuhidan2011
  ,author = {Shuhaida Shuhidan and Margaret Hamilton and Daryl D'Souza }
  ,title  = { Understanding Novice Programmer Difficulties via Guided Learning }
  ,journal= {ITiCSE}
  ,address= {Darmstadt, Germany}
  ,publisher= {ACM}
  ,year   = {2011}
}

% My last paper
@inproceedings{castleberry2016effect,
  title={The Effect of Question Ordering Using Bloom's Taxonomy in an e-Learning Environment},
  author={Castleberry, Dennis and Brandt, Steven R},
  booktitle={International Conference on Computer Science Education Innovation \& Technology (CSEIT). Proceedings},
  pages={22},
  year={2016},
  organization={Global Science and Technology Forum}
}

