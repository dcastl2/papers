\section{Experiments Supporting the Utility of the Taxonomy}

\subsection{Experiment 1 Design}. 

Our first experiment tested to see if there is a performance difference between
computer-based assessment and paper-based assessment when questions are ordered
by Bloom level.  Our hypothesis was that students taking the computer-based
assessment would fare better than those taking the paper-based assessment
because of the immediate feedback offered by the computer-based assessment.

We designed a test of 10 questions (2 concepts, each concept having questions
over 5 Bloom levels) to give to students\footnote{All questions may be viewed
at: https://steam.cct.lsu.edu/assessment/}. The questions were of
multiple-choice and short-answer format.  There were two knowledge,
comprehension, application, analysis, and evaluation questions.  No synthesis
questions were asked because of the constrained formats allowed by the
computer-based testing framework.  An interrater reliability of 90\% was
determined by two independent raters (the authors), both computer science
educators, who assessed the Bloom levels of the questions.  After a point of
disagreement about an evaluation-level problem, the test was adjusted to yield
an 100\% interrater reliability.

We designed a test of 10 questions (with 2 concepts, each concept having
questions over 5 Bloom levels) to give to students. The concepts tested were on
recursion and binary trees, and were written to be language-independent.  The
questions were of multiple-choice and short-answer format.  There were two
knowledge, comprehension, application, analysis, and evaluation questions.  No
synthesis questions were asked because of the constrained formats allowed by
the computer-based testing framework.  An interrater reliability of 90\% was
determined by two independent raters, both computer science educators, who
assessed the Bloom levels of the questions.  After a point of disagreement
about an evaluation-level problem, the test was adjusted to yield an 100\%
interrater reliability.

Evaluation-level problems are resistant to multiple-choice and short-answer
formats because of the nature of the category.  Our approach was to use
multiple-choice questions of the ``choose the best answer'' format, in which
there are many proposed uses of a concept or language construct, but one stands
out as the most sensible from the standpoint of experts. 

The test resembled a quiz that might be given in the normal course of teaching
the class. At the end of the quiz, the question ``how satisfied were you with
the (paper/computer)-based medium?'' was asked to gauge satisfaction
differences as well.

For this experiment, volunteers were recruited from a Java programming class
for introductory computer science students.  All students volunteered; candy
was offered as an incentive for all the experiments.  We split the classroom
into two groups, matched based on their current grade in the course.  We gave
the control group the paper quiz, and the experimental group the computer-based
quiz.

An answer was scored as totally correct if it coincided exactly with the
solution, and otherwise scored as incorrect. Correct answers were encoded with
1, incorrect with 0.  A composite score was derived by summing these scores
per-student.

\subsection{Experiment 2 Design}

The second experiment tested the effect of ordering the questions by Bloom
level.  For this experiment we designed another test of 10 questions (2
concepts, each concept having questions over 5 Bloom levels).  This test also
covered recursion and binary trees. In the control condition, questions were
given in forward Bloom-level order.  In the experimental condition, they were
given in reverse order.

For both Experiments 2 and 3, participants were recruited in the same manner;
however for this experiment a C++ class which followed the same conceptual
track was also added to the pool. Matched-pairs were assigned to each group
bsaed on their current grade in each course.  The test was constructed and
interrater reliability gauged in the same manner, and the test was also scored
in the same manner.

\subsection{Experiment 3 Design}. 

The third experiment tested the effect of intervening questions on the
performance of later questions in the assessment.  Our hypothesis was that
overall performance would be improved if incorrect answers triggered the
addition of new {\em intervention} questions from a lower Bloom level.

Experiment 3 participants were recruited from a different class. The test was
this time language-dependent (MATLAB) and tested mastery of control structures,
in particular for-loops.  In the control condition, the control group was given
an assessment of 10 questions, with 2 questions per the first five Bloom
levels. The experimental group was given an adaptive measure.  If at any point
a student answered a question incorrectly, then a question at the next lowest
level was given.  This applied to all levels except knowledge.  So for example
if a student answered an application-level question incorrectly, a
comprehension-level question (related to the application-level question) was
scheduled before another application-level question of the same type.  The
experimental group thus had a a maximum of 4 additional questions asked for a
total possible 14-question test.

\section{Analysis and Results}

\subsection{Experiment 1 Analysis} 

The experimental condition ($N$=27 $M$=6.21) did in fact show a higher mean
score than the control condition ($N$=27 $M$=5.23) in overall performance.
Statistical significance was tested with a one-tailed two-sample matched-pairs
Student's t-test on the composite score. The result indicated a statistically
significant difference ($t$=2.024, $p$=0.048).  The experimental condition
($M$=4.93) showed a higher mean score than the control condition ($M$=4.38) in
satisfaction as well; a similar t-test was done and was marginally
statistically significant ($t$=1.7753, $p$=0.082).  


\subsection{Experiment 2 Analysis} 

The experimental condition ($N$=48, $M$=4.94) showed a higher mean score than
the control condition ($N$=48, $M$=4.31).  Statistical significance was tested
with a one-tailed parametric Student's t-test on the composite score.  The
result indicated a statistically significant difference ($t$=2.13, $p$=0.036).

\subsection{Experiment 3 Analysis}

To tell the immediate effect of the intervention questions, one-tailed
parametric Student's t-test on the composite score of questions starting after
the first intervention question was done. It was hypothesized that the
experimental condition would perform better on the remainder of the test. The
experimental group ($N$=45, $M$=6.98) outperformed the control group ($N$=45,
$M$=6.23).  The result indicated a marginally statistically significant
difference ($t$=1.7082, $p$=0.092).

\subsection{Limitations of These Experiments}

A few validity concerns are to be pointed out. No random order was given in
Experiment 2 because of a lack of available subjects; hence it cannot be
inferred that forward order is no different from random order. The
experimenters (paper authors) designed the tests. The number of items per test
was small to allow for a conservative testing time.  For Experiment 1,
confounding variables (those other than immediate feedback) may have played a
role in test-taking because the test-taking media were different.


\section{Experiments Supporting the Reconciliation of the Taxonomy with IRT}

\subsection{Experiment 1 Design}

In the first experiment, a selection of 16 multiple-choice questions on topics
in operating systems concepts were asked to a sample group of students (N=54)
in an undergraduate-level operating systems class.  Each question had 5 choices
($\gamma$=.2). These questions were tagged with Bloom levels, and interrater
reliability was calculated between two independent raters with experience in
curricular and course design.  Questions were tagged with hypothesized
difficulty levels, which were targeted to be inversely proportional to the
Bloom levels.  Hence the question author intended for the knowledge questions
to be difficult relative to analysis questions, while still maintaining the
appropriate Bloom category.

\subsection{Experiment 2 Design}

To test this modified theory, another experiment was conducted.  The purpose of
this experiment was to explore the relationship among the probability predicted
by unmodified IRT, the probability predicted by modified IRT, and the
correctness of the actual response.  It was hypothesized that modified IRT
should produce a more accurate output than unmodified due to its account of
dependency relationships.

As in the first experiment, a selection of 16 multiple-choice questions on
topics in operating systems concepts were asked to the same sample group of
students (N=54) in the same manner ($\gamma$=.2).  The questions covered 4
concepts, and each concept had 4 questions, each of different Bloom levels
(Knowledge, Comprehension, Application, Analysis), where each question of a
Bloom level higher than Knowledge was dependent on a question of an
immediately-lower Bloom level (e.g. the Application question was dependent on
the Comprehension question of the same concept).  These questions were asked in
forward Bloom order.  The hypothesized difficulties were held equal across
Bloom levels.

\subsection{Experiment 3 Design}

One question remains: can $p_i^{mod}$ be used in some meaningful way?  Suppose
we wish to maximize the probability that a student answers an Analysis question
correctly, and that this Analysis question depends on an Application question.
For that matter, it may depend on more questions, each of which having its own
degree of relatedness to the depender.

We thus posit a more general form of modified IRT which takes into account
multiple dependencies using factor analysis \cite{kim1978}.  From a factor
analysis, it is possible to obtain the proportion of variance explained by each
dependency by squaring the factor loading.  A factor analysis was performed on
the dependent question using dependees as factors to obtain these proportions
of variance explained.

Those proportions of variance could then be used as part of a more generalized
modification to IRT:

\begin{align*}
  p_i(\theta) &=& (1- (\displaystyle\sum_j v_j)) & \Big(\gamma_i + \frac{1-\gamma_i}{1+e^{\alpha_i(\theta-\beta_i)}}\Big) \\
              &+& \displaystyle\sum_j (v_j)   & \Big(\mathbf{sgn}(r)x_{sj}+\frac{1-\mathbf{sgn}(r)}{2}\Big)
\end{align*}

In this generalized formula, $v_j$ refers to the proportion of variance
explained by the dependee $j$ (the squared factor loading for $j$).  Therefore
the sum of these is the total explained variance; one minus the sum is the
total unexplained variance.  Intuitively the probability is proportional to the
amount of variance explained by a dependee.  On the one hand, if one dependee
explains 90\% of the total variance, and the student answers this dependee
correctly, the probability should depend heavily on whether or not the dependee
was correct.  On the other hand, if the total unexplained variance is 1, this
would imply the dependees are not dependees at all.  In this case, the
probability defaults to the item response theory calculation, which uses the
best available information (trait ability and question parameters).

In our experiment to test the utility of this formula, we first required a
small problem set with one depender and multiple dependees.  The experiment
was intended to test whether or not the asking of highly-dependent questions
increased mean scores.

For this we first required proportions of variance between depender and
dependees.  We created 4 questions, where the 4th question had three dependees.
We then asked some of our students (N=22) to answer all of these questions to
obtain proportions of variance, which are listed below.  

\begin{center}
\begin{tabular}{|l|l|l|l|l|}
                                  \hline
      &   q1 &  q2  & q3 & q4  \\ \hline
$v_j$ &  .02 & .23 & .30 &     \\ \hline
 M    &  .67 & .66 & .43 & .34 \\ \hline
\end{tabular}
\end{center}

The others would receive a combination of questions.  Each student was randomly
assigned to one of eight groups.  Four students received each combination of
three questions, such that 4 received the depender outright, 12 received one of
the three dependees (4 each) before the depender, 12 received two of the three
dependees (6 each) prior, and 4 received all three dependees.  The purpose of
this was to test the effect of exposure of each dependee.  Thus each dependee
question appeared in half of the students' assessments.  The total number of
respondents for the depender was 32, and the total number of respondents for
each dependee was 16.  

\section{Analysis and Results}

\subsection{Experiment 1 Analysis}

Interrater reliability on Bloom levels
was calculated to be 93\%.  Means for each Bloom level are reported below.
Clearly

\begin{center}
\begin{tabular}{|l|l|l|l|l|}
                                                                 \hline
      &   Knowledge & Comprehension & Appplication & Analysis \\ \hline
 $M$  &  .42 & .53 & .67 & .82  \\ \hline
\end{tabular}
\end{center}

One-tailed Student t-tests were performed on the three consecutive pairs of
means.  The mean for comprehension was statistically significantly greater than
the mean for knowledge ($t=2.00$), the same held true for application and
comprehension ($t=2.33$), and analysis and application ($t=2.64$).  These
statistics, in conjunction with the interrater reliability for the assignment
of Bloom levels, support the view that Bloom level and difficulty are
distinguishable.

In light of this data, a modification of IRT was sought in order to account for
the dependency relationships that existed among the questions.  Intuitively,
this modification of $p_i$ should possess the properties that: (a) if a
question $x_j$ is unrelated, $p_i$ should be unaffected; and (b) if $x_j$ is
fully related, $p_i$ should be equal to the response value for $x_j$. 

The degree of relationship may be expressed with Pearson's $r$, where
$r_{ij}=0$ indicates no relationship between $x_i$ and $x_j$ and $r=1$
indicates a full positive correlation.  An even more viable metric is the
coefficient of determination $r^2$, which indicates the proportion of variance
in the depender which is attributable to the dependee.  Intuitively, it is this
proportion of the probability which should be associated with the correctness
of $j$.  The remainder may be left to item response theory.  In the event of a
negative correlation, the response should be opposite.


\begin{align*}
  p_i(\theta) &=& (1-r^2) & \Big(\gamma_i + \frac{1-\gamma_i}{1+e^{\alpha_i(\theta-\beta_i)}}\Big) \\
              &+& (r^2)   & \Big(\mathbf{sgn}(r)x_{sj}+\frac{1-\mathbf{sgn}(r)}{2}\Big)
\end{align*}

The above formula satisfies our needs.  The \textbf{sgn} function gives the
sign of $r$.  For $r=1$, $p_i$ defaults to the correctness of the dependee's
answer (0 or 1, depending on whether or not the dependee was answered
correctly);  and for $r=-1$, $p_i$ assumes the flip of the dependee's answer.
For $r=0$, $p_i$ assumes the form of original IRT.  

In the MLE estimation of $\theta$, we would omit question $j$, since it is
included already in the above formula.


\subsection{Experiment 2 Analysis}

Interrater reliability on Bloom levels was once again calculated to be 93\%. 

\begin{center}
\begin{tabular}{|l|l|l|l|l|}
                                                                 \hline
      &   Knowledge & Comprehension & Appplication & Analysis \\ \hline
 $M$  &  .78 & .62 & .48 & .41  \\ \hline
\end{tabular}
\end{center}


To test the hypothesis that modified IRT leads to an increase in accuracy,
a t-test on the increase in accuracy of probability was conducted at the
.05 significance level.

First the probabilities $p_i^{irt}(\theta)$ (for original IRT) and
$p_i^{mod}(\theta)$ (for modified IRT) were calculated.  Note that each of
these formulas require $\theta$. The value for $\theta$ was calculated
per-student using the MLE method on the first data set.  To determine whether
or not $p_i^{mod}(\theta)$ was on the whole more accurate than $p_i^{irt}$, the
result

\[
  \Delta p_{si} = (2x_{si}-1) (p_i^{mod} - p_i^{irt})
\]

was calculated. Here $p_{si}$ represents the probability that student $s$ will
answer item $i$ correctly.  Then $\Delta p_{si}$ represents the increase in
accuracy of the probability; for example, it is positive if $p_i^{mod} >
p_i^{irt}$ and $x_{si} = 1$.

Finally, to test the effectiveness of $p_i^{mod}$, a one-tailed Student's
t-test was calculated on $\Delta p_{si}$ for the dependee questions (N=648).
The test revealed that $\Delta p_{si}$ is statistically significantly greater
than zero ($t=2.05$), implying that $p_i^{mod}$ indeed produces a more accurate
estimate than IRT alone. 

As the data reveals, the proportion of students who passed a question did not
agree with the hypothesized difficulties.  It was suspected that this may be
due to the effect of item dependencies, which the next experiment investigates
in further detail.

\subsection{Experiment 3 Analysis}  

The analysis of this data required examining the relationship between dependent
response and dependee exposure.

Let $E_{sj}$ be 1 if student $s$ was exposed to $j$, 0 otherwise.  Then, our
hypothesis was that exposure of dependee questions should result in an increase
in the probability that the student answers the depender correctly, in
proportion to the amount of variance explained by the dependee.  Below in the
table is the number of times the depender was answered correctly given
exposure to a set of dependees.  Though the data set is small, there does
appear to be a trend in the data.  A clearer trend can be observed by summing
the number of correct responses per exposure to a given question.

\begin{center}
\begin{tabular}{|l|l|l|l|l|}
\hline
dependees    & \#correct &  percent & \#correct & percent \\ \hline
none         &  1  & .25  &1&  .06 \\ \hline
q1           &  1  & .25  &8&  .50 \\ \hline
q2           &  2  & .50  &10& .62 \\ \hline
q3           &  3  & .75  &11& .69 \\ \hline
q1, q2       &  2  & .50  && \\ \hline
q1, q3       &  2  & .50  && \\ \hline
q2, q3       &  3  & .75  && \\ \hline
q1, q2, q3   &  3  & .75  && \\ \hline
\end{tabular}
\end{center}

The table shows the percent of students who answered the depender correctly
given a particular set of questions (left), as well as the percent of students
who answered the depender correctly given exposure to a given question (right). 
It is interesting to note that to an extent, the proportions of variance
explained by a given question appear to be co-related to the percentage of
students who answered the depender correctly.  However, the numbers are too
small to draw a definitively conclusion.

Due to the low $N$ in Experiment 3, the results do not present
\emph{conclusive} evidence that the intervention of the dependee questions was
responsible.  However, there is enough preliminary evidence to warrant further
investigation.  If there is an effect, it is suspected that exposure to the
dependee problem solution by the ITS plays a larger role than simply the
question itself.


