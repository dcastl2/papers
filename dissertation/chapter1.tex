
\section{Bloom's Taxonomy}

Bloom's cognitive taxonomy organizes questions into levels depending on the
cognitive functions required of the answerer.  The levels are: knowledge,
comprehension, application, analysis, evaluation, and synthesis.  A brief
overview is given below, with definitions and examples of questions covering
the concept of for-loops:

\begin{itemize}

\item \textbf{Knowledge}. Recalling factual information.  \emph{What is a
for-loop?}

\item \textbf{Comprehension}. Assigning meaning to information.  \emph{What
does the example for-loop output? (Give example.)}

\item \textbf{Application}. Applying a rule to a specific instance.  \emph{How
can the update statement of the loop be changed to print only even numbers?}

\item \textbf{Analysis}. Breaking information into parts and exploring
relationships.  \emph{What would happen if the update statement decremented
instead of incremented the counter?}

\item \textbf{Evaluation}. Judging the use of knowledge or the validity of an
argument.  \emph{Which is better for reading user input: a for-loop or a
while-loop? Why?}

\item \textbf{Synthesis}. Utilizing knowledge to create a new solution to
satisfy a goal.  \emph{Write a for-loop to print only even numbers up to ten.}

\end{itemize}

Each category depends on the cognitive functions used in the previous category.
That is, comprehension requires knowledge, application requires comprehension,
and so forth.  Furthermore the mastery of one of the levels is with respect to
a given concept.  A student may be able to synthesize solutions to problems
dealing with expressions, but may not possess knowledge of equations, and thus
could not solve problems involving equations.

The utility of Bloom's taxonomy lies in its ability to pinpoint the underlying
cause of the student's problem-solving impasses \cite{shuhidan2011}.  Suppose a
test of mastery of loops is given with the comprehension question ``What does
such-and-such loop output?'' is given, and the student reaches an impasse.  If
the question ``What are the three expressions of the loop and what do they
do?'' is asked and the student does not know, the impasse can be attributed to
a lack of knowledge about loops.  If the student does know about the loop
expressions but still cannot answer, one might instead attribute it to a
comprehension difficulty \emph{as such}; which might be remedied by giving some
examples to build intuition, then continuing to test at the comprehension
level.

Educators may have an intuitive notion of how to do this, but Bloom's taxonomy
gives the ability to examine the impact of questions scientifically.  By
identifying the tested concept and the Bloom level of exercises, one can then
form hypotheses about student responses to questions.  


\subsection{The Interpretation in Computer Science}

Bloom's taxonomy has been proven to be useful at the undergraduate level, and
particularly in the field of software engineering \cite{britto2015,
mahmood2014}.  It has seen success in program comprehension
\cite{buckley2003}, where the asking of comprehension questions fosters code
reading \cite{losada2008}. In addition, it has been useful for pinpointing the
difficulties of novice programmers in a guided learning approach
\cite{shuhidan2011}.  It been used to identify a marked preference for
higher-level problems for those able to solve them \cite{bruyn2011}
\cite{goel2004}.  At least two experiments have shown the effect of item
ordering on performance \cite{newman1988effect,castleberry2016effect}, and the
taxonomy has even been applied to create ratings of courses based on the
average Bloom levels of tasks and questions in the course
\cite{oliver2004course}.

In spite of all this, there is an ongoing debate regarding the applicability of
Bloom's taxonomy to computer science \cite{johnson2006bloom,
fuller2007developing, thompson2008bloom}.  The crux of this debate centers
around the interpretation of Bloom levels: not only how questions map to Bloom
levels, but also regarding the progression of Bloom levels over the span of a
course or curriculum.  A tacit assumption in much of the research is that Bloom
levels equate to difficulty levels.  The current research seeks to open a
dialogue about this interpretation in particular.


\subsection{Assumptions of this Work}

\section{Item Response Theory}

We now introduce a mature assessment theory known as Item Response Theory, an
alternative to Classical Test Theory (CCT).  Whereas Classical Test Theory
assigns a student a grade based on the student's position in a distribution of
composite test scores, Item Response Theory accounts for item difficulty,
item discrimination, the probability of guessing the question correctly.

One of the main appeals of IRT is its incorporation of item difficulty.
As we will see later, this is pertinent to our interpretation of Bloom
levels.  

According to Item Response Theory (IRT), the probability $p_i$ that a student
answers correctly the $i^{th}$ question on a test, is given by:

\[
  p_i(\theta) = \gamma + \frac{1-\gamma_i}{1+e^{\alpha_i(\theta-\beta_i)}}
\]

where:

\begin{itemize} 

 \item $\alpha$ is the item discrimination, or how well the item can
 distinguish students of varying trait ability;

 \item $\beta$ is the question difficulty, an initial estimate of which can be
 obtained from the proportion of students with average trait ability who pass
 the question;

 \item $\gamma$ is the probability of guessing the answer correctly,
 which for $n$-choice questions is $1/n$;

 \item and $\theta$ is the \emph{trait ability} of the student, or the
 student's particular ability to answer that question correctly;

\end{itemize} 

Trait ability in IRT can be obtained using the maximum likelihood estimation
(MLE) method, which finds a maximum-likelihood estimate of $\theta$ by testing
a range of $\theta$ values with the IRT formula \cite{baker2004}.  The utility
of IRT is that it requires fewer questions to gauge trait ability due to its
account of other factors ($\alpha, \beta, \gamma$). It is thus reputed to be a
more mature means of assessment than CTT.

Until now, the fusion of IRT and Bloom's taxonomy has only existed in the
literature as a possibility \cite{sitthisak}.  This work seeks to reconcile
the two by offering a compatible interpretation of Bloom's taxonomy.

\subsection{Evaluation of Trait Ability}

\section{Factor Analysis}

\subsection{Confirmatory vs. Exploratory}
\subsection{Principal Axis Factoring}

\section{Previous Research}

